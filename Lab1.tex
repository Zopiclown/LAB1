\documentclass[times,5p, twocolumn]{elsarticle}
\usepackage[spanish, es-tabla]{babel}
%\usepackage[latin1]{inputenc}   % Lengua latina
\usepackage{inputenc}   % Lengua latin
\usepackage{amsmath,amsthm,amssymb}
%\usepackage{flushend}           % Para igualar las columnas de la última página
\usepackage{graphicx}
%\usepackage[pdftex]{graphicx}
\usepackage{epstopdf}                           % Convertir imagenes eps a pdf compilando con pdfLatex
\epstopdfsetup{outdir=./}
\usepackage[final]{pdfpages}                    % Paquete para incluir archivos PDF como si traslaparas hojas
\DeclareGraphicsExtensions{.eps}
\usepackage{listings}
\usepackage{color}
\usepackage[figuresright]{rotating}
\usepackage{indentfirst}
\usepackage{parskip}
\spanishdecimal{.}
\usepackage{pslatex}
\usepackage{listings, color}
\usepackage{color}
\usepackage{xcolor}
\definecolor{dkgreen}{rgb}{0,0.6,0}
\definecolor{dred}{rgb}{0.545,0,0}
\definecolor{dblue}{rgb}{0,0,0.545}
\definecolor{lgrey}{rgb}{0.9,0.9,0.9}
\definecolor{gray}{rgb}{0.4,0.4,0.4}
\definecolor{darkblue}{rgb}{0.0,0.0,0.6}
\lstdefinelanguage{ListCpp}{
	basicstyle=\footnotesize \ttfamily \color{black} \bfseries,   
	breakatwhitespace=false,       
	breaklines=true,               
	captionpos=t,                   
	commentstyle=\color{dkgreen},   
	deletekeywords={...},          
	escapeinside={\%*}{*)},                  
	frame=single,                  
	language=C++,                
	keywordstyle=\color{orange},  
	morekeywords={analogRead,analogWrite,getSpeed,getADC,MCP23S17,
		MCP3208,AD8804,SpiMotorsV4,SPI,Expander32,MCP,Slave,setQEI,
		begin,pinMode,digitalWrite,digitalRead,Serial,loop,setup,print,println}, 
	identifierstyle=\color{black},
	stringstyle=\color{blue},      
	numbers=left,                 
	numbersep=5pt,                  
	numberstyle=\tiny\color{black}, 
	rulecolor=\color{black},        
	showspaces=false,               
	showstringspaces=false,        
	showtabs=false,                
	stepnumber=1,                   
	tabsize=5,                     
	title=\lstname,                 
}

\begin{document}
\begin{frontmatter}
\title{Ondas Estacionarias: Frecuencia Fundamental y Armónicos\\ Laboratorio n°1}

\author{Vanesa Boivin}
\ead{vanesa.boivin.f@mail.pucv.cl}

% Si necesita especificar una URL para el autor, consulte 'ejemplo latex RIAI.tex'
% \ead[url]{www.autor1.es}

\author{Juan Ríos}
\ead{juan.rios.m@mail.pucv.cl}

\author{Tomás Troncoso}
\ead{tomas.troncoso.v@mail.pucv.cl}

\author{Matías Zúñiga}
\ead{asesor@correo.correo.correo}

\cortext[cor1]{Marcos Antonio Sepúlveda, Profesor}

\address{Instituto de Física, PUCV}

\begin{abstract}
250 palabras // al termianr el informe

Todo reporte de práctica tiene como propósito dejar evidencia escrita del trabajo experimental que se realiza en los laboratorios, para cumplir con este fin un reporte debe contener puntos clave que permitan al lector duplicar y/o modificar los  resultados. En la presente guía para la redacción y elaboración de reporte de práctica se muestra lo mínimo necesario para llevar acabo esta tarea de manera correcta.

All practice's reports has the purpose to leave written evidence of experimental work done in laboratories, to satisfy this purpose, a report should contain key points that allow the reader to duplicate and / or modify the results. This guide presents the correctly way to prepare a practice's report with the minimum required to complete this task. 

\end{abstract}
\begin{keyword}
Ondas estacionarias \sep Frecuencia Fundamental \sep Ecuación de Onda \sep Armónicos
\end{keyword}
\end{frontmatter}

%%%%%%%%%%%%%%%%%%%%%%  Practice's report body  %%%%%%%%%%%%%%%%%%%%%%
\section{Introducción}

Se dice que Pitagoras descubrió que cuando se hace sonar a la vez dos cuerdas similares bajo la misma tensión y diferentes sólo en longitud, dan un efecto que es agradable al oído. 


En esta sección se introduce al lector en el tema, se dan ejemplos de aplicaciones similares en otros lugares del mundo o en la industria. La introducción es un apartado breve en el que se describe de una manera general un panorama del trabajo que se va a presentar. Debe presentarse de manera resumida el alcance del trabajo y un resumen del mismo.

\section{Objetivos}
\begin{itemize}
    \item Analizar el comportamiento de una onda viajera en una cuerda con extremos fijos, que es perturbada por una señal periódica externa.
\end{itemize}
\begin{itemize}
    \item Obtener el valor de la frecuencia fundamental a partir de la ecuacion de onda.
\end{itemize}
\begin{itemize}
    \item Comprender los efectos de la superposición de ondas.
\end{itemize}

\section{Planteamiento del problema}
Para el primer experimento, consideramos una cuerda con sus extremos fijos, que está recibiendo una señal de frecuencia externa, que hará oscilar a la cuerda,  entonces nuestra misión es, obtener la frecuencia fundamental de la cuerda, ya que de esta manera podemos obtener los modos normales de la cuerda. 
Luego, para el segundo experimento, grabamos el sonido de una cuerda de guitarra, de esta manera podriamos preguntarnos cual será la frecuencia fundamental de la cuerda. 


\section{Marco teórico}
Tomemos el ejemplo simple de una onda unidimensional en una cuerda. Del mismo modo se podria considerar el sonido en una dimensión contra una pared, u otra situación de naturaleza similar, pero el ejemplo de una cuerda seria suficiente para los fines que nos hemos propuesto. Para tener mejor comprension sobre las ondas unidimensionales, debemos tener claros algunos conceptos. 








\section{Materiales y métodos}
\begin{itemize}
    \item Cuerda delgada
    \item Regla
    \item Balanza
    \item Generador de frecuencia
    \item Extremos fijos para afirmar la cuerda
    \item Polea y pesos NO SE SI IMPORTA ESTA PARTE
\end{itemize}

\section{Desarrollo}

En esta sección se debe presentar de manera ordenada los pasos que se siguieron durante la experimentación. Esta sección se fortalece mediante el uso de imágenes y una descripción secuencial de los pasos a seguir. Es importante la redacción de esta sección, pues de ella depende la reproducibilidad del experimento que se realizó.

\section{Resultados y análisis}
\subsection{Ecuaciones}
Tenemos la ecuación de onda $\nabla^2\psi(\vec r, t)=\dfrac
1{v^2}\dfrac{\partial^2\psi
(\vec r,t)}{\partial t^2} $ 


En los reportes de ingeniería es común incluir ecuaciones para describir la solución a un problema o el comportamiento de un fenómeno.

De preferencia, las ecuaciones deben ser escritas en algún software editor de textos matemáticos para que mantengan estética, y por lo tanto, sean fáciles de interpretar.

Las ecuaciones deben encontrarse enumeradas, de manera que sea fácil ubicarlas al seguir el procedimiento descrito en el texto.

Por ejemplo, la ecuación \ref{eqtn001} corresponde a la línea recta cuando se conocen dos puntos, $(x_1,y_1)$ y $(x_2,y_2)$, que pertenecen a la misma. A diferencia de las imágenes, las ecuaciones no necesitan una extrema descripción y tampoco deben ser todas citadas en el texto, sin embargo, es necesario que lleven una congruencia que pueda seguir el lector.

\begin{equation}
y - y_1  = \frac{{y_2  - y_1 }}{{x_2  - x_1 }}\left( {x - x_1 } \right)
\label{eqtn001}
\end{equation}

Cuando se presenta un algoritmo o metodología, no es necesario citar todo el procedimiento a seguir, para evitar esto puede citarse la fuente de la que fue extraída el algoritmo o desarrollo matemático, y en el trabajo únicamente incluir las ecuaciones más significativas. Por ejemplo, de la ecuación \ref{eq59} a la ecuación \ref{eq65} se muestra el algoritmo para implementar el \emph{método de identificación de mínimos cuadrados recursivos}.

\subsubsection{Algoritmo MMC recursivo\\}

\begin{equation}
g(t + 1) = C(t)z(t + 1)
\label{eq59}
\end{equation}

\begin{equation}
\alpha ^2 (t + 1) = \phi ^2  + z^T (t + 1)g(t + 1)
\label{eq60}
\end{equation}

\begin{equation}
\hat e(t + 1) = y(t + 1) - \hat P^T (t)z(t + 1)
\label{eq61}
\end{equation}

\begin{equation}
\hat P(t + 1) = \hat P(t) + \frac{1}{{\alpha ^2 (t + 1)}}g(t + 1)\hat e(t + 1)
\label{eq62}
\end{equation}

\begin{equation}
v(t + 1) = 1 + \phi ^2 v(t)
\label{eq63}
\end{equation}

\begin{equation}
\sigma ^2 (t + 1) = \frac{\phi }{{v(t + 1)}}\left[ {v(t)\hat \sigma ^2 (t) + \frac{1}{{\alpha ^2 (t + 1)}}\hat e(t + 1)\hat e^T (t + 1)} \right]
\label{eq64}
\end{equation}

\begin{equation}
C(t + 1) = \frac{1}{{\phi ^2 }}\left[ {C(t) - \frac{1}{{\alpha ^2 (t + 1)}}g(t + 1)g^T (t + 1)} \right]
\label{eq65}
\end{equation}

El método puede verificarse en \cite{aguado}. Como se puede observar se presentan las ecuaciones en las que se basa el método, sin embargo es necesario explicar el desarrollo matemático que nos permite formular estas ecuaciones, ya que este ya ha sido desarrollado en la cita bibliográfica que se menciona.  



En esta sección se presentan los resultados obtenidos en la experimentación. Es fuertemente reforzada mediante el uso de gráficas obtenidas de los resultados y la presentación de datos en tablas. El análisis se refiere a la comparación entre los resultados que se obtuvieron y los resultados que se esperan, también puede reforzarse esta sección con el uso de gráficas comparativas y datos estadísticos comunes (desviación estándar, error cuadrático medio, promedio, media, mediana, moda, entre otras)., sin embargo no siempre es necesario mostrar estos datos.


\section{Elementos auxiliares del reporte}
\input{./imagenes.tex}
\input{./tablas.tex}
\input{./ecuaciones.tex

\section{Conclusiones}

later

\cite{Yeadon} Las conclusiones que se obtienen se deben relacionar directamente con los resultados del análisis que previamente se realizó. Las conclusiones deben ser respecto a los objetivos que se plantearon en la introducción y deben plasmar el aprendizaje que obtuvo por el alumno de la práctica que desarrolló. Se deben evitar la redacción redundante, por lo que se recomienda utilizar párrafos independientes para cada idea que expresa una conclusión particular e independiente.
















\subsection{Código}

\begin{lstlisting}[language=ListCpp,
caption={Código \LaTeX para insertar código en C/C++},
label = {codigo_1}]

[y,Fs]=audioread('G.wav')
[y,Fs]=audioread('G.wav'); %Leer el audio
sound(y)
%% 
y = y/max(abs(y));%Normalizar el audio
%Transformada de fourier
transformada = abs(fft(y)); % Aplicar la transformada de Fourier
L = length(transformada);% El tamaño del vector de la transformada.

espectro = transformada(1:L/2);% Cuando hacemos la transformada hay algo
                               %que es la trasnformada bilateral, espejeada
                               %en el eje x
espectro = espectro/max(abs(espectro)); % Normalizo.
frecuencias = Fs*(1:L/2)/L;% Vector de frecuencias.

n=length(y);% Tamaño del vector de audio
t = n/Fs;%Tiempo que dura el audio
Ts = 1/Fs;%Periodo de muestreo
tiempo= 0:Ts:(t-Ts); %Vector de tiempo
figure;
grafica1=subplot(2,1,1);
plot(tiempo,y,'b')
xlabel('Tiempo(s)');
ylabel('Amplitud');

grafica2=subplot(2,1,2);
plot(frecuencias,espectro,'r')
xlabel('Frecuencia(Hz)');
ylabel('Amplitud');
set(gca,'Ytick',0.1);
ylim([0, max(espectro)]);
xlim([0, 3000]);
grid(grafica2,'on');


\end{lstlisting}

\bibliographystyle{El feyman vol 1
Hecht
el de metodos matematicos}
\bibliography{reporte}	

\end{document} 